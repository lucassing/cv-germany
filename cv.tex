%%%%%%%%%%%%%%%%%%%%%%%%%%%%%%%%%%%%%%%%%
% cv-friggeri-x 1.0 (01/01/2016)
% XeLaTeX Template
%
% Based on:
% Friggeri Resume/CV
% Version 1.2 (3/5/15)
%
% Original author:
% Adrien Friggeri (adrien@friggeri.net)
% https://github.com/afriggeri/CV
% 
% Modified by:
% Nadorrano
% https://github.com/Nadorrano/cv-friggeri-x
%
% License:
% MIT
%
% Important notes:
% This template needs to be compiled with XeLaTeX and the bibliography,
% if used, needs to be compiled with biber rather than bibtex.
%
%%%%%%%%%%%%%%%%%%%%%%%%%%%%%%%%%%%%%%%%%

\documentclass[a4paper,nocolors]{cv-friggeri-x}
% Add `a4paper` to set a4 paper size
% Add `nocolors` to remove colors from the document
% Add `lightheader` to change the dark background of the header to white

\usepackage{marvosym} % needed to print glyphs for email, cell phone etc.
\addbibresource{bibliography.bib} % Specify the bibliography file to include publications
\begin{document}

\header{Lucas Martin }{Sing}{Python Software Engineer} % Your name and current job title/field

%----------------------------------------------------------------------------------------
%	SIDEBAR SECTION
%----------------------------------------------------------------------------------------

\begin{aside} % In the aside, each new line forces a line break
\section{contact}
\pin \hfill Kahlweg 9
35398 Gießen-Allendorf
Germany
~
{\Large\textcolor{gray}{\Mobilefone}} \hfill +49 (177) 7527 030
\hfill \href{mailto:singlucasmartin@gmail.com}{\footnotesize singlucasmartin@gmail.com}
~
\flogo \hfill \href{http://facebook.com/lucas.sing}{fb://lucas.sing}
\llogo \hfill \href{https://www.linkedin.com/in/lucassinggg}{in://lucassinggg}
~
\section{languages}
\begin{description}
  \item[Spanish] \hfill Native speaker
  \item[English] \hfill Fluent
  \item[German] \hfill Intermediate (B1)
\end{description}
\section{stack}
\begin{description} 
  \item[Python] \hfill \filleddot \filleddot \filleddot \filleddot \emptydot
  \item[Node.Js] \hfill \filleddot \filleddot \filleddot \emptydot \emptydot
  \item[Django] \hfill \filleddot \filleddot \filleddot \filleddot \filleddot 
  \item[React.Js] \hfill \filleddot \filleddot \filleddot \emptydot \emptydot
  \item[AWS] \hfill \filleddot \filleddot \filleddot \emptydot \emptydot
  \item[GCloud] \hfill \filleddot \filleddot \filleddot \emptydot \emptydot
  \item[Docker] \hfill \filleddot \filleddot \filleddot \filleddot \emptydot
  \item[Terraform] \hfill \filleddot \emptydot \emptydot \emptydot \emptydot
\end{description}
\end{aside}

%----------------------------------------------------------------------------------------
%	EDUCATION SECTION
%----------------------------------------------------------------------------------------

\section{education}

\begin{entrylist}

%------------------------------------------------

\entry
{2012--2019}
{Engineer’s Degree {\normalfont in Electronics}}
{Universidad Tecnológica Nacional, Argentina}
{Graduated with a focus on electronics engineering.}

%------------------------------------------------

\entry
{2017--2018}
{Internship {\normalfont in Electronics Engineering}}
{Ruhr University Bochum, Germany}
{Scholarships from DAAD (German Academic Exchange Service) for students in engineering sciences.}

%------------------------------------------------
\end{entrylist}

%----------------------------------------------------------------------------------------
%	WORK EXPERIENCE SECTION
%----------------------------------------------------------------------------------------

\section{experience}

\subsection{Full Time}

\begin{entrylist}

%------------------------------------------------

\entryexperience
{Aug 2024--Now}
{Globant - Gannett USA Today Network }
{Germany · Remote}
{As part of the Gannett User Platform team for Gannett USA Today Network, I am responsible for developing and maintaining backend services and APIs that power personalized user experiences across more than 20 active and maintained projects. Our tech stack includes Python, Django, DRF, PostgreSQL, and tools like GitHub Actions and Grafana for monitoring, deployment, and continuous integration.}
{Python (Programming Language) · Django · Gcloud · Terraform · Github Actions · Django REST Framework · pytest · Node.js · PostgreSQL · Grafana}

%------------------------------------------------

\entryexperience
{2023--2024}
{Prezi - Payments }
{Germany · Remote}
{Designed and built payments platform features with Django and React. Integrated payment gateways and third-party APIs (Zuora, Avalara, Allpago) for secure transactions. Developed Grafana dashboards for monitoring. Automated regression testing using Selenium. Diagnosed payment processing issues and improved codebase performance.}
{Django · React.js · Amazon Web Services (AWS) · PostgreSQL · GitHub Actions · Jenkins · Selenium · Grafana · Sentry · Payment and tax APIs (Zuora, Avalara, Allpago, etc.)}
%------------------------------------------------

\entryexperience
{2023--2024}
{Prezi - Porter}
{Germany · Remote}
{As one of three developers behind Porter by Prezi, a Zoom-integrated tool designed to enhance meeting facilitation and participation, I contributed to making team sessions more productive, collaborative, and engaging. I was responsible for designing, developing, and maintaining features within a small development team. Additionally, I integrated the Zoom App SDK to enhance real-time collaboration and meeting facilitation. Working closely with the team, I helped refine the user experience and improve functionality based on user feedback.}
{Heroku, Firebase, Express.js, TypeScript, React.js, Zoom App SDK}
%------------------------------------------------

\entryexperience
{2023--2024}
{ValueWorks - Fullstack Developer}
{Germany · Remote}
{As part of the development team, I contributed to building a cutting-edge executive management system that leveraged AI to enhance decision-making, planning, reporting, and execution. My work spanned both frontend and backend development, including end-to-end feature implementation and third-party system integrations. The team designed, developed, and maintained the platform using Vue.js and Django REST, ensuring a seamless user experience. We implemented and optimized database solutions with PostgreSQL for efficient data storage and retrieval while leveraging Redis for caching and performance enhancements. Additionally, we integrated third-party systems and services using Azure and collaborated closely to build and refine end-to-end product features.}
{Vue.js, Django REST Framework, PostgreSQL, Redis, Microsoft Azure}
%------------------------------------------------

\entryexperience
{2023--2024}
{Justus Liebig Universitat - Fullstack Python-js Developer}
{Gießen, Hesse, Germany · Hybrid}
{As part of the development team, I contributed to building an HPTLC Chromatography Instrument from scratch with a full-stack approach. The team designed and implemented backend, frontend, and firmware components while also overseeing a group of students and mentoring junior developers. We developed the backend using Python (Django) and SQLite, while the frontend was built with JavaScript and jQuery to create an intuitive user interface. The firmware, written in C/C++, ensured seamless hardware-software integration. Additionally, we integrated OpenCV for advanced image processing tasks. Throughout the project, we supervised and mentored five bachelor’s and master’s students and trained two additional developers for future development and maintenance.}
{SQLite, Django REST Framework, jQuery, C++, OpenCV, GitHub Actions, Docker}

\end{entrylist}

%----------------------------------------------------------------------------------------
%	AWARDS SECTION
%----------------------------------------------------------------------------------------

\section{awards}

\begin{entrylist}

%------------------------------------------------

\entry
{2011}
{Postgraduate Scholarship}
{School of Business, The University of California}
{Awarded to the top student in their final year of a Bachelors degree. Mastered the art of filing accurate TPS reports.}

%------------------------------------------------

\end{entrylist}

%----------------------------------------------------------------------------------------
%	COMMUNICATION SKILLS SECTION
%----------------------------------------------------------------------------------------

\section{communication skills}

\begin{entrylist}

%------------------------------------------------

\entry
{2011}
{Oral Presentation}
{California Business Conference}
{Presented the research I conducted for my Masters of Commerce degree.}

%------------------------------------------------

\entry
{2010}
{Poster}
{Annual Business Conference, Oregon}
{As part of the course work for BUS320, I created a poster analyzing several local businesses and presented this at a conference.}

%------------------------------------------------

\end{entrylist}

%----------------------------------------------------------------------------------------
%	INTERESTS SECTION
%----------------------------------------------------------------------------------------

\section{interests}

\textbf{professional:} data analysis, company profiling, risk analysis, economics, web design, web app creation, software design, marketing \textbf{personal:} piano, chess, cooking, dancing, running

%----------------------------------------------------------------------------------------
%	PUBLICATIONS SECTION
%----------------------------------------------------------------------------------------

\section{publications}

\printbibsection{article}{article in peer-reviewed journal} % Print all articles from the bibliography

\printbibsection{book}{books} % Print all books from the bibliography

\begin{refsection} % This is a custom heading for those references marked as "inproceedings" but not containing "keyword=france"
\nocite{*}
\printbibliography[sorting=chronological, type=inproceedings, title={international peer-reviewed conferences/proceedings}, notkeyword={france}, heading=bibheading]
\end{refsection}

\begin{refsection} % This is a custom heading for those references marked as "inproceedings" and containing "keyword=france"
\nocite{*}
\printbibliography[sorting=chronological, type=inproceedings, title={local peer-reviewed conferences/proceedings}, keyword={france}, heading=bibheading]
\end{refsection}

\printbibsection{misc}{other publications} % Print all miscellaneous entries from the bibliography

\printbibsection{report}{research reports} % Print all research reports from the bibliography

%----------------------------------------------------------------------------------------

\href{http://www.germanitjobs.com}{German IT Jobs} helps you to land your dream programming job in Germany

\end{document}
